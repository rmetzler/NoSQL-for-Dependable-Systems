% This is LLNCS.DEM the demonstration file of
% the LaTeX macro package from Springer-Verlag
% for Lecture Notes in Computer Science,
% version 2.4 for LaTeX2e as of 16. April 2010
%
\documentclass{llncs}
%
\usepackage{makeidx}  % allows for indexgeneration
\usepackage{hyperref}
%
\begin{document}
%
\frontmatter          % for the preliminaries
%
\pagestyle{headings}  % switches on printing of running heads
\addtocmark{NOSQL in Dependable Systems} % additional mark in the TOC
%
\mainmatter              % start of the contributions
%
\title{NOSQL in Dependable Systems}
%
\titlerunning{NOSQL in Dependable Systems}  % abbreviated title (for running head)
%                                     also used for the TOC unless
%                                     \toctitle is used
%
\author{Richard Metzler \and Jan Sch\"utze}
%
\authorrunning{Ivar Ekeland et al.} % abbreviated author list (for running head)
%
%%%% list of authors for the TOC (use if author list has to be modified)
\tocauthor{Richard Metzler and Jan Sch\"utze}
%
\institute{Universit\"at Potsdam, Am Neuen Palais 10, 14469 Potsdam, Germany,\\
\email{presse@uni-potsdam.de},\\ 
\texttt{http://uni-potsdam.de}
}

\maketitle              % typeset the title of the contribution

\begin{abstract}
The fault model for very large e-commerce websites like Amazon is fundamentally
different from standard websites. These websites loose money when the aren't
available (or just slow) for potential customers but can't risk to loose any
data. The data has to be replicated between databases but traditional RDBMSs
may not fit. This paper discusses some of the better known NoSQL software
products available today.
\keywords{fault tolerance, cap theorem, nosql, dynamo, riak, cassandra, mongodb, couchdb}
\end{abstract}
