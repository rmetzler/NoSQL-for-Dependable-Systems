\section{Fault Model}


On very large e-commerce platforms like Amazon that serve
millions of customers using thousands of servers located
in different datacenters around the world reliability is one
of the most important requirements because even the slightest
outage has significant financial consequences. \cite{dynamo} 

There is a particular need for storage technologies that are 
always available. For example, the customer's shopping cart
has to be always accessible for writes. But on the other side failures 
are the normal case rather than the exception. Disks fail,
 the network experiences partitioning and
whole data centers could become potentially unavailable because of
natural disasters like hurricanes.

What our big e-commerce websites need is a datastore that is always
read and write enabled, even in presence of network partitions.
Data must be replicated across multiple datacenters and these
datacenters may be located hundreds of kilometers away from each
other and even on different continents.

\section{Replication}

\emph{Replication} is one of the fundamental ideas for fault
tolerant systems. But replicating data across datacenters located
several hundreds of kilometers away from each other takes time.
Using a traditional RDBMS with ACID style transactions to replicate
data in a distributed transaction may be slow and not very
scalable. Synchronous atomic updates would not be tolerant towards
network partitions.

Asynchronous updates can't be atomic, but they are potentially more
resistant in case of network partitioning as these are usually
transient faults.

\subsection{Split Brain}

In distributed systems the interconnect between nodes is a weak
spot. If it is broken, nodes are split into partitions unable to
communicate and thus unable to share state. This scenario is called
\emph{split brain}. Nodes in split brain scenarios must be
prevented from producing inconsistant state and one method to
prevent inconsistency is the quorum consensus.

\subsection{Quorum}

As the system replica managers in different partitions cannot
communicate with each other, the subgroup of replica managers
within each partition must be able to decide independently whether
they are allowed to carry out operations. A quorum is a subgroup of
replica managers whose size gives it the right to carry out
operations. \cite{Coulouris} One possible criteria for a
quorum may be having a majority. Any other partition would be
smaller than the majority partition and as a consequence only the
majority partition would be the quorum. Another possible quorum
criteria could be the availability of a \emph{quorum device}. The
partition that is able to access the quorum device is allowed to
carry out operations.

\section{Brewer's CAP Theorem}

In 2000 Eric Brewer at this time chief scientist of Inktomi hold a
keynote at the ``Principles of Distributed Computing'' conference.
He presented his assumption that was later proved in \cite{brewers-conjecture} 
stating that atomic data consistency, high availability (i.e.
performance) and network partition tolerance can't be achieved all
together at any given time and you may get only two of these
properties for every distributed operation. This is called the \emph{CAP
Theorem} after the acronym for \textbf{C}onsistency,
\textbf{A}vailability and \textbf{P}artition tolerance.

Because it is impossible to prevent network partitions in large
networks the decision has to be between high availability and data
consistency. As stated, large e-commerce websites usually go for
high availability and trade consistency for that.

\section{Eventual Consistent}

Werner Vogels, CTO at Amazon, presented in his article
``Eventually Consistent'' \cite{vogels}
his idea of data being not consistent through atomic transactions
but only eventually consistent. By trading ACID's atomicy and
consistency for performance and partition tolerance it is possible
to increase the response time and fault tolerance of websites. The
database replications may not be fully consistent but a customer
wouldn't usually experience any inconsistencies.

Vogeles defines the \emph{inconsistency window} as
\emph{``The period between the update and the moment when it is guaranteed that any observer will always see the updated value.''}

\section{N / W / R Replica Configuration}

Vogels introduces the reader to a short notation for replication
configuration for \emph{quorum} like systems:

\begin{itemize}
\item
  \textbf{N} is the number of nodes, that store replicas of the data
\item
  \textbf{W} is the number of replicas that acknowledge a write
  operation
\item
  \textbf{R} is the number of replicas contacted in a read operation
\end{itemize}
To avoid ties in failover scenarios usually an odd number is picked
for N.

With these numbers it is guaranteed \emph{strong consistency} if
following condition holds: N \textless{} W + R . This is because the
set of replicas for writing and reading the data overlap. If the
replica configuration holds the condition  N \textgreater{}= W + R
it only guaranties weak or eventual consistency.

It is possible to deduce different attributes from these
configuration properties. Consistency over all nodes is reached if
W = N . Read optimized systems will use R = 1, while write optimized
systems use W = 1. A RDBMS is typically configured with \{N = 2, W = 2, R = 1\} 
\newline while \{N = 3, W = 2, R = 2\} is a common configuration for fault
tolerant systems.

Dynamo is able to run in application specific N / W / R
configuration. This helps Dynamo to recover from transient and
permanent failures. Apache Cassandra also implements this as application
specific while with Riak it is possible to specify this per operation.

\section{Ring topology}

Cassandra and Riak are both heavily inspired by Amazon's Dynamo and organize
replicating nodes in a ring topology like Dynamo does. Also Cloudant's BigCouch
\cite{bigcouch} for CouchDB is closely modeled after Dynamo
and features a ring to replicate CouchDB instances.

N consecutive nodes on the ring form one replica set for a certain
data partition. Replica sets overlap as every node is the main node
for a data partition and the replica for one or more other data partitions.
%By distributing the nodes in the physical space a better fault tolerance should be obtained.

\section{Products}

There are several NoSQL systems available. We focused on 4 of the
major ones:

\begin{itemize}
\item
  Riak (document oriented)
\item
  Cassandra (column oriented)
\item
  CouchDB (document oriented)
\item
  MongoDB (document oriented)
\end{itemize}
In terms of the CAP theorem: Riak, Cassandra and CouchDB provide
availability and partition tolerance. MongoDB on the other hand
provides consistency and partition tolerance.

We installed those on multiple virtual machines, connected them
with each other and ran some very simple tests to figure out how
they behave in case of a fault.

\subsection{Riak 0.11.0}

Riak is created and maintained by Basho (a company founded by
Ex-Akamai employees) and the de facto open source reference
implementation of the dynamo paper. It is released under the terms
of Apache License 2.0 .

Riak is a document oriented key value store and also supports links
between them. Documents are stored in so called buckets. Riak is
written entirely in Erlang and has an HTTP interface to read and
write data.

More information about Riak may be found at
\url{https://wiki.basho.com/display/RIAK/Riak}.

\subsubsection{Replication Config}

One of Riaks features is the variable configuration of N W R:

\begin{itemize}
\item
  N can vary for each bucket
\item
  R \& W can vary for each operation (read/write/delete)
\end{itemize}
There are also additional quorum properties for 
\emph{durable writes to disk} and \emph{delete} operation.


\subsection{Cassandra 0.6.3}

Cassandra is a column oriented distributed database system written
in Java. It was created by Facebook and donated to the Apache
Foundation. Cassandra is released under the Apache License 2.0.

Cassandra's architecture is inspired by Amazon's Dynamo and the
Google BigTable.

When developing applications with Cassandra the developer need to
configure the collumns before Cassandra can be started. Thus
Cassandra is not schema less like one may expect from the term
NoSQL. Once configured you can search and order the documents by
these columns.

More information about Cassandra may be found at the official
website at \url{http://cassandra.apache.org/} or at the wiki at
\url{http://wiki.apache.org/cassandra/FrontPage}.

\subsection{MongoDB 1.4.3}

MongoDB is a document oriented distributed database system by
10gen. It is available under the terms of GNU Affero General Public
License.

It uses a custom TCP protocal (BSON) and is written in C++.

More information about MongoDB is available at the official website
at \url{http://www.mongodb.org/} or at the wiki at
\url{http://www.mongodb.org/display/DOCS/Home}.

\subsubsection{Replication}

There is Master/Slave replication in MongoDB available. In this
case one can read from the slaves and the master, but write only
into the master.

Additionally there are Replica Pairs available. When using Replica
Pairs only one of two nodes is the master at any time. Read and
write is only possible on the master of the Replica Pair.

When one node of the Replica Pair fails the other one is made
the new master. This decision of who is the master is done by a
configured arbiter (Quorum Device).

The MongoDB Team is currently working on Replica Sets, which are
meant to allow more then 2 machines to be part of the replication
configuration.

\subsubsection{Crash}

In case of a MongoDB crash, the entire database must be reindexed
again. According to David Mytton's blogpost
\url{http://blog.boxedice.com/2010/02/28/notes-from-a-production-mongodb-deployment/}
this takes up to 72 hours for 664.000.000 database entries.

That's why MongoDB has an increased MTTR (Mean Time To Repair).

\subsection{CouchDB 0.11.0}

CouchDB is a document oriented database written in Erlang with
support for JavaScript views. CouchDB is an Apache Project.
Access to the data is made available via an HTTP interface by
exchanging JSON objects.

More information about CouchDB is available at the official website
at \url{http://couchdb.apache.org} or at the wiki at
\url{http://wiki.apache.org/couchdb/FrontPage}.

\section{Experiments}

To test the fault tolerance features of the distributed database
systems, we focused on the behavior in case of network splits and
synchronization after adding new nodes.

For this purpose we set up the virtual machines ``Alice'' and
``Bob''. Both run a vanilla Debian Squeeze release in Virtual Box.
For the Cassandra experiments we added a third identical machine
called ``Charly''.

\subsection{Experiment 1}

The first experiment is meant to show what happens if a new node
joins the distributed database.

For this purpose we set up the node Alice and pushed 1000 data
records into Alice. Then Bob joined the network. To check if Bob
already had all data we frequently tried to read the 1000th entry
from Bob. If this was possible we assumed that Bob was in sync or
at least capable to answer in a consistent way.

Results (Replicating to a new node):

\begin{itemize}
\item
  Riak: 1 second
\item
  Cassandra (3 nodes): 20 seconds
\item
  CouchDB: 1 second
\item
  MongoDb: 2 seconds
\end{itemize}
\subsection{Experiment 2}

To test how the distributed database system is able to manage a
network split, we set up the second experiment.

The two nodes Alice and Bob were synchronized and connected. Then
we deactivated Bob's network and wrote 1000 data records into
Alice. Because of the network partition it is impossible for Bob to
have the fresh data. We turned on the network and timed how long it
takes for Bob to receive all data entries (by querying for the
1000th entry).

Results (replication after network split):

\begin{itemize}
\item
  Riak: 6 seconds
\item
  Cassandra (3 nodes): 20 seconds
\item
  CouchDB: 8 seconds
\item
  MongoDb: Failed, because reading from Slave returns an error
\end{itemize}
When configured as replica pair it is not possible to read from the
MongoDB Slave.

\subsection{Experiment 2b}

Since we had MongoDB as Replica Pair in the experiment 2, it was
impossible to read from the slave. That's why we made an experiment
2b with a slightly different configuration.

We set up Alice and Bob as Replica Pair. Another MongoDB instance
``Charly'' was used as arbiter (Quorum Device). MongoDB chose the
first one (Alice) to be the master and Bob the slave. Then we
pushed 1000 data records into Alice.

We stopped Alice in three different ways:

\begin{itemize}
\item
  by removing the network connection
\item
  by stopping it gracefully
\item
  by using \texttt{kill -9}
\end{itemize}
After that we timed how long Bob needs to recognize that Alice has
disappeared and Bob become master.

Result: It took 1 second for Bob to become Master and thus allowing
the client to read from Bob. We noticed that stopping the node and \texttt{kill -9} worked great. But
Bob did not notice the network split if we just removed the network
connection. We assume that this is because the virtual machine does
not send any interupt like a physical network interface would
waiting for the TCP connection to time out.

%\section{Sources}
%
%\begin{itemize}
%\item
%  Eric Brewer: ``Towards Robust Distributed Systems''
%  \url{http://www.cs.berkeley.edu/~brewer/cs262b-2004/PODC-keynote.pdf}
%\item
%  Gilbert, Lynch:
%  ``Brewer‘s Conjecture and the Feasibility of Consistent, Available, Partition-Tolerant Web Services''
%\item
%  Werner Vogels: ``Eventual Consistent''
%\item
%  W. Vogels et all: "Dynamo: Amazon's highly Available Key-Value
%  Store
%\item
%  Lakshman, Malik:
%  ``Cassandra - A Decentralized Structured Storage System''
%\item
%  David Mytton: ``Notes from a production MongoDB deployment''
%  \url{http://blog.boxedice.com/2010/02/28/notes-from-a-production-mongodb-deployment/}
%\item
%  Coulouris et al: ``Distributed Systems. Concepts and Design''
%\end{itemize}